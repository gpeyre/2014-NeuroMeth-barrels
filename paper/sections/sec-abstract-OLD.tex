% !TEX root = ../neurosciences-barrels.tex

\begin{abstract} % 241 words (250 max)
The rodent's barrel cortex is a widely used model to study the cortical processing of tactile sensory information. It is remarkable by the cytoarchitecture of its layer IV, which contains distinguishable structural units called barrels that can be considered as anatomical landmarks of the functional columnar organization of the cerebral cortex. These barrels are juxtaposed in a topographic manner, which reflects, in the posteromedial barrel subfield, the spatial arrangement of the whiskers on the snout of the animal.  When studying sensory processing in the barrel cortex either with electrophysiological or imaging methods, it is therefore of great interest to superimpose the recorded activity onto the underlying barrel topography. However, this anatomo-functional mapping is usually time-consuming and difficult to accomplish accurately with conventional manual methods. This article presents an automated workflow to perform the registration of histological slices of the mouse barrel cortex followed by the reconstruction of the barrel map from the registered slices. The registration of two successive slices is obtained by computing a rigid transform to align sets of detected blood vessels cross-sections. This is achieved by using a robust variant of the classical iterative closest point method. A single fused image of the barrel field is then generated by computing a nonlinear merging of the gradients from the registered images. This novel anatomo-functional mapping tool provides a flexible interface for the user with few parameters to tune. Its application is exemplified here for voltage sensitive dye imaging experiments.
\end{abstract}

\begin{keyword}
barrel cortex \sep histological sections \sep blood vessels \sep registration \sep robust iterative closest point \sep gradient domain fusion \sep voltage sensitive dye imaging.
%% keywords here, in the form: keyword \sep keyword
%% PACS codes here, in the form: 
% \PACS ??? \sep ???
%% MSC codes here, in the form: \MSC code \sep code
%% or \MSC[2008] code \sep code (2000 is the default)
\end{keyword}