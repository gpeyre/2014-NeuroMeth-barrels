% !TEX root = ../neurosciences-barrels.tex

\section{Feature Detection with Normalized Cross-Cor\-re\-la\-tion}
\label{app-crosscorrel}

We consider a set of Gaussian templates $\{g_k\}_{k \in \Kk}$, 
\eq{
	g_k(x) = \frac{1}{Z_k}\exp\pa{ -\frac{\norm{x}^2}{2 \si_k^2} }
}
where $Z_k$ is a constant ensuring a normalization $\sum_x g_k(x)^2 = 1$.
The standard deviations $\si_k$ are chosen equally spaced in the range $[0,\si_{\max}]$ (note that $\si_k=0$ corresponds to a Dirac, supported on a single pixel). 

Given a template $g_k$, we denote its support as 
\eq{
	I_k = \enscond{ x }{ g_k(x) > \eta }
}
where $\eta = 10^{-3}$ is a small tolerance threshold. The normalized cross correlation of a section $S$ with the template $g_k$ is then defined as
\eq{
	\NCC_k(S)(x) = \frac{ \sum_{y \in I_k} S(x+y) g_k(y)  }{
		\pa{ \sum_{y \in I_k} S(x+y)^2 }^{1/2}
	}.
}
The normalized cross correlation with the whole set of filters is the maximum of all the correlations
\eql{\label{eq-ncc-def}
	\NCC(S)(x) = \umax{k \in \Kk} \NCC_k(S)(x).
}
A large value of $\NCC(S)(x)$ indicates that a vessel is likely to be present at pixel $x$. In this case, the value $k=k(x)$ of the maximum appearing in~\eqref{eq-ncc-def}, i.e. such that $\NCC(S)(x) =  \NCC_k(S)(x)$, indicates that the radius of this vessel is approximately $\si_k$.  

% \todo{In the previous definition, the mean was subtracted from the image. This looks weird, no ?}
