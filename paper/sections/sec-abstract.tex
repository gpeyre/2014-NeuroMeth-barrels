% !TEX root = ../neurosciences-barrels.tex

\begin{abstract} % 241 words (250 max)
\textit{Highlights:}\\
-- Here is a new tool to map functional data onto the barrel cortex structure. \\
-- It realigns histological slices and reconstructs the barrel map in 2-D. \\
-- Slice realignment by rigid transformations is computed using detected blood vessels. \\
-- Barrel map reconstruction is obtained by gradient fusion. \\
-- Its application is exemplified for voltage sensitive dye imaging experiments. \\
\textit{Background:}
The rodent barrel cortex is a widely used model to study the cortical processing of tactile sensory information. It is notable by the cytoarchitecture of its layer IV, which contains distinguishable structural units called barrels that can be considered as anatomical landmarks of the functional columnar organization of the cerebral cortex. To study sensory integration in the barrel cortex it is therefore essential to map recorded functional data onto the underlying barrel topography, which can be reconstructed from the post hoc alignment of tangential brain slices stained for cytochrome oxidase. \\
\textit{New Method:} 
This article presents an automated workflow to perform the registration of histological slices of the barrel cortex followed by the 2-D reconstruction of the barrel map from the registered slices. The registration of two successive slices is obtained by computing a rigid transformation to align sets of detected blood vessel cross-sections. This is achieved by using a robust variant of the classical iterative closest point method. A single fused image of the barrel field is then generated by computing a nonlinear merging of the gradients from the registered images. \\
\textit{Comparison with Existing Methods:}
This novel anatomo-functional mapping tool leads to a substantial gain in time and precision compared to conventional manual methods. It provides a flexible interface for the user with only a few parameters to tune.
\textit{Conclusions:}
We demonstrate here the usefulness of the method for voltage sensitive dye imaging of the mouse barrel cortex. The method could also benefit other experimental approaches and model species.
%\bigskip\noindent\textit{Highlights:} \\
%-- Here is a new tool to map functional data onto the barrel cortex structure. \\
%-- It reconstructs the barrel map from realigned histological slices. \\
%-- Slices realignment by rigid transforms is computed using detected blood vessels. \\
%-- Barrel map reconstruction is obtained by gradient fusion. \\
%-- Its application is exemplified for voltage sensitive dye imaging experiments. \\
\end{abstract}


\begin{keyword}
barrel cortex \sep histological sections \sep blood vessels \sep registration \sep robust iterative closest point \sep gradient domain fusion \sep voltage sensitive dye imaging.
%% keywords here, in the form: keyword \sep keyword
%% PACS codes here, in the form: 
% \PACS ??? \sep ???
%% MSC codes here, in the form: \MSC code \sep code
%% or \MSC[2008] code \sep code (2000 is the default)
\end{keyword}